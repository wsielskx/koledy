\section{Anioł pasterzom mówił}{
 \stanza{
 	Anioł pasterzom mówił:
 	Chrystus się wam narodził
 	W Betlejem, nie bardzo podłym mieście.
 	Narodził się w ubóstwie
 	Pan wszego stworzenia.
 }
 \stanza{
 	Chcąc się dowiedzieć tego
 	Poselstwa wesołego,
 	Bieżeli do Betlejem skwapliwie.
 	Znaleźli dziecię w żłobie,
 	Maryję z Józefem.
 }
 \stanza{
 	Taki Pan chwały wielkiej,
 	Uniżył się Wysoki,
 	Pałacu kosztownego żadnego
 	Nie miał zbudowanego
 	Pan wszego stworzenia.
 }
 \stanza{
 	O dziwne narodzenie,
 	Nigdy niewysłowione!
 	Poczęła Panna Syna w czystości,
 	Porodziła w całości
 	Panieństwa swojego.
 }
 \stanza{
 	Już się ono spełniło,
 	Co pod figurą było:
 	Arona różdżka ona zielona
 	Stała się nam kwitnąca
 	I owoc rodząca.
 }
 \stanza{
 	Słuchajcież Boga Ojca,
 	Jako wam Go zaleca:
 	Ten ci jest Syn najmilszy, jedyny,
 	W raju wam obiecany,
 	Tego wy słuchajcie.
 }
 \stanza{
 	Bogu bądź cześć i chwała,
 	Która byś nie ustała,
 	Jako Ojcu, tak i Jego Synowi
 	I Świętemu Duchowi,
 	W Trójcy jedynemu.
}}
\section{Cicha noc}{
 \stanza{
 	Cicha noc, święta noc!
 	Pokój niesie ludziom wszem,
 	A u żłóbka Matka Święta
 	Czuwa sama uśmiechnięta
 	Nad Dzieciątka snem,
 	Nad Dzieciątka snem.
 }
 \stanza{    
 	Cicha noc, święta noc!
 	Pastuszkowie od swych trzód
 	Śpieszą wielce zadziwieni
 	Za anielskich głosem pieni,
 	Gdzie się spełnił cud,
 	Gdzie się spełnił cud.
 }
 \stanza{    
 	Cicha noc, święta noc!
 	Narodzony Boży Syn,
 	Pan wielkiego majestatu
 	Niesie dziś całemu światu
 	Odkupienie win,
 	Odkupienie win.
 }
 \stanza{
 	Cicha noc, święta noc,
 	Jakiż w tobie dzisiaj cud,
 	W Betlejem dziecina święta
 	Wznosi w górę swe rączęta
 	Błogosławi lud,
 	Błogosławi lud.
}}
\section{Gdy się Chrystus rodzi}{
 \stanza{
 	Gdy się Chrystus rodzi i na świat przychodzi
 	Ciemna noc w jasności promienistej brodzi.
 }
 \chorus{
 	Aniołowie się radują, pod niebiosy wyśpiewują:
 	Gloria, gloria, gloria in excelsis Deo
 }
 \stanza{
 	Mówią do pasterzy, którzy trzód swych strzegli,
 	Aby do Betlejem czem prędzej pobiegli,
 }
 \chorus{
 	Bo się narodził Zbawiciel, Wszego Świata odkupiciel
 	Gloria, gloria, gloria in excelsis Deo
 }
 \stanza{
 	O niebieskie Duchy i posłowie nieba,
 	Powiedzcież wyraźnie, co nam czynić trzeba
 }
 \chorus{
 	Bo my nic nie pojmujemy, ledwo od strachu żyjemy
 	Gloria, gloria, gloria in excelsis Deo
 }
 \stanza{
 	Idźcież do Betlejem, gdzie Dziecię zrodzone,
 	W pieluszki powite, w żłobie położone.
 }
 \chorus{
 	Oddajcie Mu pokłon Boski, on osłodzi Wasze troski
 	Gloria, gloria, gloria in excelsis Deo
 }
 \stanza{
 	A gdy pastuszkowie wszystko zrozumieli
 	zaraz do Betlejem śpieszno pobieżeli
 }
 \chorus{
 	I zupełnie tak zastali, jak anieli im zeznali
 	Gloria, gloria, gloria in excelsis Deo
 }
 \stanza{
 	A stanąwszy w miejscu pełni zadumienia,
 	Iż się Bóg tak zniżył do swego stworzenia,
 }
 \chorus{
 	Padli przed Nim na kolana i uczcili swego Pana:
 	Gloria, gloria, gloria in excelsis Deo
 }
 \stanza{
 	Nareszcie, gdy pokłon Panu już oddali
 	Z wielką wesołością do swych trzód wracali,
 }
 \chorus{
 	Że się stali być godnymi Boga widzieć na tej ziemi.
 	Gloria, gloria, gloria in excelsis Deo
 }
}
\section{Do szopy, hej pasterze}{
 \stanza{
 	Do szopy, hej pasterze,
 	Do szopy, bo tam cud,
 	Syn Boży w żłobie leży,
 	By zbawić ludzki ród.
 }
 \chorus{
 	Śpiewajcie Aniołowie,
 	Pasterze grajcież Mu,
 	Kłaniajcie się królowie,
 	Nie zbudźcie Go ze snu!
 }
 \stanza{
 	Pobiegli pastuszkowie
 	Ze swymi dary tam,
 	Oddali pokłon korny,
 	Bo to ich Bóg i Pan.
 }
 \chorus{
 	Śpiewajcie Aniołowie...
 }
 \stanza{
 	O Boże niepojęty,
 	Któż pojmie miłość Twą?
 	Na sianie wśród bydlęty
 	Masz tron i służbę swą.
 }
 \chorus{
 	Śpiewajcie Aniołowie...
 }
 \stanza{
 	Padnijmy na kolana,
 	To Dziecię to nasz Bóg,
 	Uczcijmy niebios Pana,
 	Miłości złóżmy dług!
 }
 \chorus{
 	Śpiewajcie Aniołowie...
 }
 \stanza{
 	Jezuniu mój najsłodszy,
 	Oddaję Tobie się,
 	O skarbie mój najdroższy,
 	Racz wziąć na własność mnie.
 }
 \chorus{
 	Śpiewajcie Aniołowie...
 }
 \stanza{
 	Do szopy, hej pasterze,
 	Do szopy wszyscy wraz,
 	Syn Boży w żłobie leży,
 	Więc spieszcie póki czas!
 }
 \chorus{
 	Śpiewajcie Aniołowie...
 }
}
\section{Narodził się Jezus Chrystus}{
	\stanza{
		Narodził się Jezus Chrystus, bądźmy weseli,
		Chwałę Mu na wysokości nucą Anieli:
	}
	\chorus{
		Gloria, gloria, in excelsis Deo!
		Gloria, gloria, in excelsis Deo!
	}
	\stanza{
		Na kolana wół i osioł przed Nim klękają,
		Jego swoim Stworzycielem, Panem uznają
				
	}
	\stanza{
		Pastuszkowie przybiegają na znak im dany,
		Cześć oddają i witają Pana nad pany.
	}
	\stanza{
		Trzej Królowie przyjechali z wielkimi dary,
		złoto, mirra i kadzidło, oto ofiary.
	}
	\stanza{
		I my także chwałę dajmy Dzieciątku temu,
		jako Panu nieba, ziemi, Zbawcy naszemu.
	}
}
\section{Lulajże Jezuniu}{
	\stanza{
		Lulajże, Jezuniu, moja perełko,
		Lulaj, ulubione me pieścidełko.
	}
	\chorus{
		Lulajże, Jezuniu, lulajże, lulaj,
		A Ty Go, Matulu, w płaczu utulaj.
	}
	\stanza{
		Zamknijże znużone płaczem powieczki,
		Utulże zemdlone łkaniem usteczki
	}
	\stanza{
		Dam ja Jezusowi słodkich jagódek,
		Pójdę z nim w Matuli serca ogródek.
	}
	\stanza{
		Dam ja Jezusowi z chlebem masełka,
		Włożę ja kukiełkę w jego jasełka.
	}
	\stanza{
		Lulajże, piękniuchny mój aniołeczku,
		Lulajże, maluchny świata kwiateczku.
	}
	\stanza{
		Matuniu kochana, już odchodzimy,
		Małemu Dzieciątku przyśpiewujemy.
	}
	\stanza{
		Cyt cyt cyt, już zaśnie małe Dzieciątko,
		Patrz jeno, jak to śpi niby kurczątko.
	}
	\stanza{
		Cyt cyt cyt, wszyscy się spać zabierajcie,
		Mojego Dzieciątka nie przebudzajcie.
	}
}
\section{W żłobie leży}{
	\stanza{
		W żłobie leży, któż pobieży
		Kolędować Małemu
		Jezusowi Chrystusowi,
		Dziś nam narodzonemu?
	}
	\chorus{
		Pastuszkowie, przybywajcie,
		Jemu wdzięcznie przygrywajcie,
		Jako Panu naszemu.
	}
	\stanza{
		My zaś sami z piosneczkami
		Za wami pośpieszymy,
		A tam Tego Maleńkiego
		Niech wszyscy zobaczymy:
	}
	\chorus{
		Jak ubogo narodzony,
		Płacze w stajni położony,
		Więc Go dziś ucieszymy.
	}
	\stanza{
		Naprzód tedy niechaj wszędy
		Zabrzmi świat w wesołości,
		Że posłany jest nam dany
		Emmanuel w niskości.
	}
	\chorus{
		Jego tedy przywitajmy,
		Z Aniołami zaśpiewajmy:
		Chwała na wysokości!
	}
	\stanza{
		Witaj Panie, cóż się stanie,
		Że rozkosze niebieskie
		Opuściłeś, a zstąpiłeś
		Na te niskości ziemskie?
	}
	\chorus{
		Miłość moja to sprawiła,
		Że człowieka wywyższyła
		Pod nieba Empirejskie.
	}
}
\section{Przybieżeli do Betlejem pasterze}{
\stanza{
    Przybieżeli do Betlejem pasterze,
    Grając skocznie Dzieciąteczku na lirze.
}
\chorus{
	Chwała na wysokości,
	Chwała na wysokości,
	A pokój na ziemi.
}
\stanza{
    Oddawali swe ukłony w pokorze
    Tobie z serca ochotnego, o Boże!
}
\stanza{
    Anioł Pański sam ogłosił te dziwy,
    Których oni nie słyszeli, jak żywi.
}
\stanza{
    Dziwili się napowietrznej muzyce
    I myśleli, co to będzie za Dziecię?
}
\stanza{
    Oto mu się wół i osioł kłaniają,
    Trzej królowie podarunki oddają.
}
\stanza{
    I anieli gromadami pilnują
    Panna czysta wraz z Józefem piastują.
}
\stanza{
    Poznali Go Mesyjaszem być prawym
    Narodzonym dzisiaj Panem łaskawym
}
\stanza{
    My go także Bogiem, Zbawcą już znamy
    I z całego serca wszyscy kochamy.
}
}
\section{Kaczka pstra}{
\stanza{
    Kaczka pstra, Dziatki ma,
}
\chorus{
    Siedzi sobie na kamieniu,
    Trzyma dudki na ramieniu,
    Kwa kwa kwa, pięknie gra.
}
\stanza{
    Skowronek jak dzwonek,
}
\chorus{
    Gdy do nieba się unosi,
    O kolędę pięknie prosi:
    Ćwir, ćwir, ćwir, tak prosi.
}
\stanza{
    Gęsiorek, Jędorek
}
\chorus{
    Na bębenku wybijają,
    Pana wdzięcznie wychwalają:
    Gę gę gę, gęgają.
}
\stanza{
    Słowiczek muzyczek,
}
\chorus{
    Gdy się głosem popisuje,
    Wesele światu zwiastuje:
    Ciech ciech ciech, zwiastuje.
}
\stanza{
    Wróblowie stróżowie,
}
\chorus{
    Gdy nad szopą świergotają,
    Paniąteczku spać nie dają,
    Dziw dziw dziw, nie dają.
}
\stanza{
    Czyżyczek, szczygliczek
}
\chorus{
    Na gardłeczkach jak skrzypeczkach
    Śpiewają Panu w jasłeczkach;
    Lir lir lir, w jasłeczkach.
}
}
\section{Z narodzenia Pana}{
\stanza{
Z Narodzenia Pana dzień dziś wesoły.
Wyśpiewują chwałę Bogu żywioły.
}
\chorus{
Radość ludzi wszędzie słynie.
Anioł budzi przy dolinie,
Pasterzów co paśli pod borem woły.
}
\stanza{
Wypada wśród nocy ogień z obłoku.
Dumają pasterze w takim widoku.
}
\chorus{
Każdy pyta: „Co się dzieje?
Czy nie świta, czy nie dnieje?
Skąd ta łuna bije, tak miła oku!"
}
\stanza{
Skoro pastuszkowie głos usłyszeli.
Zaraz do Betlejem prosto bieżeli.
}
\chorus{
Tam witali w żłobie Pana,
Poklękali na kolana
i oddali dary co z sobą wzięli.
}
\stanza{
Odchodzą z Betlejem pełni wesela,
Że już Bóg wysłuchał próśb Izraela,
}
\chorus{
Gdy tej nocy to widzieli.
Co prorocy widzieć chcieli,
W ciele ludzkim Boga i Zbawiciela.
}
\stanza{
I my z pastuszkami dziś się radujmy,
Chwałę z Aniołami wraz wyśpiewujmy,
}
\chorus{
Bo ten Jezus z nieba dany,
Weźmie nas między niebiany,
Tylko Go z całego serca miłujmy!
}}
\section{Oj, maluśki, maluśki}{
\stanza{
    Oj, maluśki, maluśki, maluśki kiejby rękawicka,
    Alboli tez jakoby, jakoby kawałecek smycka.
}
\chorus{
        (Ref.A): Lili-lili-lili laj, lili laj, lili laj.
        (Ref.B): Śpiewajmy i grajmy Mu, dzieciątku, małemu.
}
\stanza{
    Cy nie lepiej by Tobie, by Tobie siedzieć było w niebie
    Wsak Twój Tatuś kochany, kochany nie wyganiał Ciebie.
}
\stanza{
    Tam pijałeś coś takie, coś takie słodkie małmazyje,
    Tu się Twoja gębusia, gębusia łez gorzkich napije.
}
\stanza{
    Tam kukiełki jadałeś, jadałeś z carnuską i miodem,
    Tu się jeno zasilać, zasilać musis samym głodem.
}
\stanza{
    Tam wciornaska wygoda, wygoda, a tu bieda wsędzie,
    Ta Ci teraz dokuca, dokuca, ta i potem będzie.
}
\stanza{
    Tam Ty miałeś pościółkę, pościółkę i miętkie piernatki,
    Tu na to Twej nie stanie, nie stanie ubozuchnej Matki.
}
\stanza{
    Tam Ci zawse słuzyły, słuzyły prześlicne janioły,
    A tu lezys sam jeden, sam jeden jako palec goły.
}
\stanza{
    Co się więc takiego, takiego Tobie, Panie stało,
    Zeć się na ten kiepski świat, kiepski świat przychodzić zachciało.
}
\stanza{
    Gdybym ja tam jako Ty, jako Ty tak królował sobie,
    Nie chciałby ja przenigdy, przenigdy w tym spocywać żłobie.
}
\stanza{
    Albo się więc, mój Panie, mój Panie, wróć do swej krainy,
    Alboć pozwól się zanieść, się zanieść do nasej chałpiny.
}
}
\section{W Dzień Bożego Narodzenia}{
\stanza{
    W Dzień Bożego Narodzenia
    Radość wszelkiego stworzenia:
    Ptaszki w górę podlatują,
    Jezusowi wyśpiewują, wyśpiewują.
}
\stanza{
    Słowik zaczyna dyszkantem,
    Szczygieł mu wtóruje altem;
    Szpak tenorem krzyknie czasem,
    A gołąbek gruchnie basem, gruchnie basem.
}
\stanza{
    A mazurek ze swym synem
    Tak świergocze za kominem;
    Cierp, cierp, cierp, cierp miły Panie,
    Póki ten mróz nie ustanie, nie ustanie.
}
\stanza{
    A żurawie w swoje nosy
    wykrzykują pod niebiosy;
    czajka w górę podlatuje,
    chwałę Bogu wyśpiewuje, wyśpiewuje.
}
\stanza{
    Gdy ptactwo Boga uczciło,
    Co żywo się rozproszyło;
    Ludziom dobry przykład dali,
    Ażeby Go uwielbiali, uwielbiali.
}
}
\section{Zagrzmiała, runęła}{
\stanza{
Zagrzmiała, runęła w Betlejem ziemia,
nie było, nie było Józefa w doma.
}
\stanza{
Kędyżeś, kędyżeś, Józefie, bywał?
W Betlejem, w Betlejem Dzieciątku śpiewał.
}
\stanza{
Wół, osioł, wół, osioł przed nim klękali,
bo swego, bo swego Stwórcę poznali.
}
\stanza{
Beczący, ryczący Panu śpiewali,
pasterze, pasterze w multanki grali.
}
\stanza{
Zmiłuj się, zmiłuj się, nasz wieczny Panie,
bez Ciebie, bez Ciebie nic się nie stanie.
}
}
\section{W tej kolędzie}{
    \stanza{
    W tej kolędzie co dziś będzie każdy się ucieszy, 
A kto co ma podarować niechaj prędko śpieszy. 
    } \stanza{
Dać dary z tej miary dla Pana małego 
By nabył go śmierci zbawienia wiecznego. 
} \stanza{
Kuba stary przyniósł dary - masła na talerzu, 
Sobek parę gołąbeczków takich jeszcze w pierzu. 
} \stanza{
Wziął Tomek gomółek i jajeczko gęsie, 
A Bartek nie miał co dać - dobre chęci niesie.
} \stanza{
 Walek sprawiał tłuste raki nie rychło z wieczora.
 Nałożywszy dwie kobiele biegł z nimi przez pola.
 } \stanza{
Aż tu strach napotkał Walka nieboraka. 
Stanęły dwa wilki niedaleko krzaka. 
} \stanza{
Gdy obaczył owe gady podskoczył wysoko. 
Wielkim strachem przestraszony wybił sobie oko.
} \stanza{
Uciekał przez krzaki. Podarł se chodaki. 
A wilcy mu targali z kobieliny raki. 
} \stanza{
Szymek wziął kozę na powróz. Prowadzi do Pana. 
Śpiewa sobie, wykrzykuje dana moja dana. 
} \stanza{
Koza się zbrykała, powróz mu urwała. 
Skoczywszy jak dzika do lasu bieżała. 
} \stanza{
On porwawszy się prędziuchno biegł za nią przez krzaki. 
Koza skacze jak szalona, spłoszyły ją ptaki. 
} \stanza{
Uchwycił za ogon trzymając co mocy, 
A koza fiknęła, podbiła mu oczy. 
} \stanza{
Stach kudłaty, chłop bogaty wziął czerwony złoty. 
Nie chciał się nikomu kłaniać, biegł prędko do szopy. 
} \stanza{
Uderzył Jurka w brzuch aż mu kiszki wzruszył, 
A Jurko go za łeb, kudły mu ususzył. 
} \stanza{
Głupi Wojtek nie wziął portek. Mówił: lecej będzie. 
Po kolędzie sperki zbierał, gdzie która nabędzie. 

} \stanza{
Biegł Wojtek bez portek po śniegu, po grudzie. 
Cieszą się, śmieją się ha ha ha ha ludzie. 
} \stanza{
Maciek biegł po śliskim lodzie, wybił sobie zęby, 
A chciawszy mleko połykać leciało mu z gęby. 
} \stanza{
Biegł prędko i upadł. Rozbił z mlekiem dzbanek. 
Smucił się żałował gdy miał ten trafunek. 
} \stanza{
Przeto wszyscy oddawajmy temu dary. 
Pan to dobry, wszystkim szczodry. Przyjmie nas do chwały. 
} \stanza{
Niech będzie, niech będzie Jezu pochwalony, 
Który jest, który jest w żłobie położony.
}
}