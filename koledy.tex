\section{Anioł pasterzom mówił}{
 \stanza{
 	Anioł pasterzom mówił:
 	Chrystus się wam narodził
 	W Betlejem, nie bardzo podłym mieście.
 	Narodził się w ubóstwie
 	Pan wszego stworzenia.
 }
 \stanza{
 	Chcąc się dowiedzieć tego
 	Poselstwa wesołego,
 	Bieżeli do Betlejem skwapliwie.
 	Znaleźli dziecię w żłobie,
 	Maryję z Józefem.
 }
 \stanza{
 	Taki Pan chwały wielkiej,
 	Uniżył się Wysoki,
 	Pałacu kosztownego żadnego
 	Nie miał zbudowanego
 	Pan wszego stworzenia.
 }
 \stanza{
 	O dziwne narodzenie,
 	Nigdy niewysłowione!
 	Poczęła Panna Syna w czystości,
 	Porodziła w całości
 	Panieństwa swojego.
 }
 \stanza{
 	Już się ono spełniło,
 	Co pod figurą było:
 	Arona różdżka ona zielona
 	Stała się nam kwitnąca
 	I owoc rodząca.
 }
 \stanza{
 	Słuchajcież Boga Ojca,
 	Jako wam Go zaleca:
 	Ten ci jest Syn najmilszy, jedyny,
 	W raju wam obiecany,
 	Tego wy słuchajcie.
 }
 \stanza{
 	Bogu bądź cześć i chwała,
 	Która byś nie ustała,
 	Jako Ojcu, tak i Jego Synowi
 	I Świętemu Duchowi,
 	W Trójcy jedynemu.
}}
\section{Cicha noc}{
 \stanza{
 	Cicha noc, święta noc!
 	Pokój niesie ludziom wszem,
 	A u żłóbka Matka Święta
 	Czuwa sama uśmiechnięta
 	Nad Dzieciątka snem,
 	Nad Dzieciątka snem.
 }
 \stanza{    
 	Cicha noc, święta noc!
 	Pastuszkowie od swych trzód
 	Śpieszą wielce zadziwieni
 	Za anielskich głosem pieni,
 	Gdzie się spełnił cud,
 	Gdzie się spełnił cud.
 }
 \stanza{    
 	Cicha noc, święta noc!
 	Narodzony Boży Syn,
 	Pan wielkiego majestatu
 	Niesie dziś całemu światu
 	Odkupienie win,
 	Odkupienie win.
 }
 \stanza{
 	Cicha noc, święta noc,
 	Jakiż w tobie dzisiaj cud,
 	W Betlejem dziecina święta
 	Wznosi w górę swe rączęta
 	Błogosławi lud,
 	Błogosławi lud.
}}
\section{Zagrzmiała, runęła}{
 \stanza{ 
 	Zagrzmiała, runęła w Betlejem ziemia,
 	nie było, nie było Józefa w doma.
 }
 \stanza{
 	Kędyżeś, kędyżeś, Józefie, bywał?
 	W Betlejem, w Betlejem Dzieciątku śpiewał.
 }
 \stanza{
 	Wół, osioł, wół, osioł przed nim klękali,
 	bo swego, bo swego Stwórcę poznali.
 }
 \stanza{
 	Beczący, ryczący Panu śpiewali,
 	pasterze, pasterze w multanki grali.
 }
 \stanza{
 	Zmiłuj się, zmiłuj się, nasz wieczny Panie,
 	bez Ciebie, bez Ciebie nic się nie stanie.
 }
}
\section{Gdy się Chrystus rodzi}{
 \stanza{
 	Gdy się Chrystus rodzi i na świat przychodzi
 	Ciemna noc w jasności promienistej brodzi.
 }
 \chorus{
 	Aniołowie się radują, pod niebiosy wyśpiewują:
 	Gloria, gloria, gloria in excelsis Deo
 }
 \stanza{
 	Mówią do pasterzy, którzy trzód swych strzegli,
 	Aby do Betlejem czem prędzej pobiegli,
 }
 \chorus{
 	Bo się narodził Zbawiciel, Wszego Świata odkupiciel
 	Gloria, gloria, gloria in excelsis Deo
 }
 \stanza{
 	O niebieskie Duchy i posłowie nieba,
 	Powiedzcież wyraźnie, co nam czynić trzeba
 }
 \chorus{
 	Bo my nic nie pojmujemy, ledwo od strachu żyjemy
 	Gloria, gloria, gloria in excelsis Deo
 }
 \stanza{
 	Idźcież do Betlejem, gdzie Dziecię zrodzone,
 	W pieluszki powite, w żłobie położone.
 }
 \chorus{
 	Oddajcie Mu pokłon Boski, on osłodzi Wasze troski
 	Gloria, gloria, gloria in excelsis Deo
 }
 \stanza{
 	A gdy pastuszkowie wszystko zrozumieli
 	zaraz do Betlejem śpieszno pobieżeli
 }
 \chorus{
 	I zupełnie tak zastali, jak anieli im zeznali
 	Gloria, gloria, gloria in excelsis Deo
 }
 \stanza{
 	A stanąwszy w miejscu pełni zadumienia,
 	Iż się Bóg tak zniżył do swego stworzenia,
 }
 \chorus{
 	Padli przed Nim na kolana i uczcili swego Pana:
 	Gloria, gloria, gloria in excelsis Deo
 }
 \stanza{
 	Nareszcie, gdy pokłon Panu już oddali
 	Z wielką wesołością do swych trzód wracali,
 }
 \chorus{
 	Że się stali być godnymi Boga widzieć na tej ziemi.
 	Gloria, gloria, gloria in excelsis Deo
 }
}
\section{Do szopy, hej pasterze}{
 \stanza{
 	Do szopy, hej pasterze,
 	Do szopy, bo tam cud,
 	Syn Boży w żłobie leży,
 	By zbawić ludzki ród.
 }
 \chorus{
 	Śpiewajcie Aniołowie,
 	Pasterze grajcież Mu,
 	Kłaniajcie się królowie,
 	Nie zbudźcie Go ze snu!
 }
 \stanza{
 	Pobiegli pastuszkowie
 	Ze swymi dary tam,
 	Oddali pokłon korny,
 	Bo to ich Bóg i Pan.
 }
 \stanza{
 	O Boże niepojęty,
 	Któż pojmie miłość Twą?
 	Na sianie wśród bydlęty
 	Masz tron i służbę swą.
 } 
 \stanza{
 	Padnijmy na kolana,
 	To Dziecię to nasz Bóg,
 	Uczcijmy niebios Pana,
 	Miłości złóżmy dług!
 }
 \stanza{
 	Jezuniu mój najsłodszy,
 	Oddaję Tobie się,
 	O skarbie mój najdroższy,
 	Racz wziąć na własność mnie.
 }
 \stanza{
 	Do szopy, hej pasterze,
 	Do szopy wszyscy wraz,
 	Syn Boży w żłobie leży,
 	Więc spieszcie póki czas!
 }
}
\section{Narodził się Jezus Chrystus}{
 \stanza{
 	Narodził się Jezus Chrystus, bądźmy weseli,
 	Chwałę Mu na wysokości nucą Anieli:
 }
 \chorus{
 	Gloria, gloria, in excelsis Deo!
 	Gloria, gloria, in excelsis Deo!
 }
 \stanza{
 	Na kolana wół i osioł przed Nim klękają,
 	Jego swoim Stworzycielem, Panem uznają
 	 					
 }
 \stanza{
 	Pastuszkowie przybiegają na znak im dany,
 	Cześć oddają i witają Pana nad pany.
 }
 \stanza{
 	Trzej Królowie przyjechali z wielkimi dary,
 	złoto, mirra i kadzidło, oto ofiary.
 }
 \stanza{
 	I my także chwałę dajmy Dzieciątku temu,
 	jako Panu nieba, ziemi, Zbawcy naszemu.
 }
}
\section{Lulajże Jezuniu}{
 \stanza{
 	Lulajże, Jezuniu, moja perełko,
 	Lulaj, ulubione me pieścidełko.
 }
 \chorus{
 	Lulajże, Jezuniu, lulajże, lulaj,
 	A Ty Go, Matulu, w płaczu utulaj.
 }
 \stanza{
 	Zamknijże znużone płaczem powieczki,
 	Utulże zemdlone łkaniem usteczki
 }
 \stanza{
 	Dam ja Jezusowi słodkich jagódek,
 	Pójdę z nim w Matuli serca ogródek.
 }
 \stanza{
 	Dam ja Jezusowi z chlebem masełka,
 	Włożę ja kukiełkę w jego jasełka.
 }
 \stanza{
 	Lulajże, piękniuchny mój aniołeczku,
 	Lulajże, maluchny świata kwiateczku.
 }
 \stanza{
 	Matuniu kochana, już odchodzimy,
 	Małemu Dzieciątku przyśpiewujemy.
 }
 \stanza{
 	Cyt cyt cyt, już zaśnie małe Dzieciątko,
 	Patrz jeno, jak to śpi niby kurczątko.
 }
 \stanza{
 	Cyt cyt cyt, wszyscy się spać zabierajcie,
 	Mojego Dzieciątka nie przebudzajcie.
 }
}
\section{W żłobie leży}{
 \stanza{
 	W żłobie leży, któż pobieży
 	Kolędować Małemu
 	Jezusowi Chrystusowi,
 	Dziś nam narodzonemu?
 }
 \chorus{
 	Pastuszkowie, przybywajcie,
 	Jemu wdzięcznie przygrywajcie,
 	Jako Panu naszemu.
 }
 \stanza{
 	My zaś sami z piosneczkami
 	Za wami pośpieszymy,
 	A tam Tego Maleńkiego
 	Niech wszyscy zobaczymy:
 }
 \chorus{
 	Jak ubogo narodzony,
 	Płacze w stajni położony,
 	Więc Go dziś ucieszymy.
 }
 \stanza{
 	Naprzód tedy niechaj wszędy
 	Zabrzmi świat w wesołości,
 	Że posłany jest nam dany
 	Emmanuel w niskości.
 }
 \chorus{
 	Jego tedy przywitajmy,
 	Z Aniołami zaśpiewajmy:
 	Chwała na wysokości!
 }
 \stanza{
 	Witaj Panie, cóż się stanie,
 	Że rozkosze niebieskie
 	Opuściłeś, a zstąpiłeś
 	Na te niskości ziemskie?
 }
 \chorus{
 	Miłość moja to sprawiła,
 	Że człowieka wywyższyła
 	Pod nieba Empirejskie.
 }
}
\section{Kaczka pstra}{
 \stanza{
 	Kaczka pstra, Dziatki ma,
 }
 \chorus{
 	Siedzi sobie na kamieniu,
 	Trzyma dudki na ramieniu,
 	Kwa kwa kwa, pięknie gra.
 }
 \stanza{
 	Skowronek jak dzwonek,
 }
 \chorus{
 	Gdy do nieba się unosi,
 	O kolędę pięknie prosi:
 	Ćwir, ćwir, ćwir, tak prosi.
 }
 \stanza{
 	Gęsiorek, Jędorek
 }
 \chorus{
 	Na bębenku wybijają,
 	Pana wdzięcznie wychwalają:
 	Gę gę gę, gęgają.
 }
 \stanza{
 	Słowiczek muzyczek,
 }
 \chorus{
 	Gdy się głosem popisuje,
 	Wesele światu zwiastuje:
 	Ciech ciech ciech, zwiastuje. 
 }
 \leavevmode\newpage
 \stanza{
 	Wróblowie stróżowie,
 }
 \chorus{
 	Gdy nad szopą świergotają,
 	Paniąteczku spać nie dają,
 	Dziw dziw dziw, nie dają.
 }
 \stanza{
 	Czyżyczek, szczygliczek
 }
 \chorus{
 	Na gardłeczkach jak skrzypeczkach
 	Śpiewają Panu w jasłeczkach;
 	Lir lir lir, w jasłeczkach.
 }
}
\section{W Dzień Bożego Narodzenia}{
 \stanza{
 	W Dzień Bożego Narodzenia
 	Radość wszelkiego stworzenia:
 	Ptaszki w górę podlatują,
 	Jezusowi wyśpiewują, wyśpiewują.
 }
 \stanza{
 	Słowik zaczyna dyszkantem,
 	Szczygieł mu wtóruje altem; 
 	Szpak tenorem krzyknie czasem, 
 	A gołąbek gruchnie basem, gruchnie basem.
 }
 \stanza{
	Wróbel, ptaszek nieboraczek,
	uziąbłszy śpiewa jak żaczek:
	dziw, dziw, dziw, dziw, dziw nad dziwy,
	narodził się Bóg prawdziwy, Bóg prawdziwy.
 }
 \stanza{
 	A mazurek ze swym synem
 	Tak świergocze za kominem;
 	Cierp, cierp, cierp, cierp miły Panie,
 	Póki ten mróz nie ustanie, nie ustanie.
 }
 \stanza{
 	A żurawie w swoje nosy
 	wykrzykują pod niebiosy;
 	czajka w górę podlatuje,
 	chwałę Bogu wyśpiewuje, wyśpiewuje.
 }
 \stanza{
 	Sroka wlazła na jedlinę,
	odarła sobie łysinę
	i choć gołe świeci czoło,
	śpiewa jednak dość wesoło, dość wesoło.
 }
 \stanza{
	Kur na grzędzie krzyczy wszędzie:
	wstańcie, ludzie, bo dzień będzie!
	Do Betlejem pospieszajcie,
	Boga w ciele oglądajcie, oglądajcie.
 }
 \stanza{
 	Gdy ptactwo Boga uczciło,
 	Co żywo się rozproszyło;
 	Ludziom dobry przykład dali,
 	Ażeby Go uwielbiali, uwielbiali.
 }
}

\section{Z narodzenia Pana}{
 \stanza{
 	Z Narodzenia Pana dzień dziś wesoły.
 	Wyśpiewują chwałę Bogu żywioły.
 }
 \chorus{
 	Radość ludzi wszędzie słynie.
 	Anioł budzi przy dolinie,
 	Pasterzów co paśli pod borem woły.
 }
 \stanza{
 	Wypada wśród nocy ogień z obłoku.
 	Dumają pasterze w takim widoku.
 }
 \chorus{
 	Każdy pyta: „Co się dzieje?
 	Czy nie świta, czy nie dnieje?
 	Skąd ta łuna bije, tak miła oku!"
 }
 \stanza{
 	Skoro pastuszkowie głos usłyszeli.
 	Zaraz do Betlejem prosto bieżeli.
 }
 \chorus{
 	Tam witali w żłobie Pana,
 	Poklękali na kolana
 	i oddali dary co z sobą wzięli.
 }
 \stanza{
 	Odchodzą z Betlejem pełni wesela,
 	Że już Bóg wysłuchał próśb Izraela,
 }
 \chorus{
 	Gdy tej nocy to widzieli.
 	Co prorocy widzieć chcieli,
 	W ciele ludzkim Boga i Zbawiciela.
 }
 \stanza{
 	I my z pastuszkami dziś się radujmy,
 	Chwałę z Aniołami wraz wyśpiewujmy,
 }
 \chorus{
 	Bo ten Jezus z nieba dany,
 	Weźmie nas między niebiany,
 	Tylko Go z całego serca miłujmy!
}}
\section{Oj, maluśki, maluśki}{
 \stanza{
 	Oj, maluśki, maluśki, maluśki kiejby rękawicka,
 	Alboli tez jakoby, jakoby kawałecek smycka.
 }
 \chorus{
 	(Ref.A): Lili-lili-lili laj, lili laj, lili laj.
 	(Ref.B): Śpiewajmy i grajmy Mu, dzieciątku, małemu.
 }
 \stanza{
 	Cy nie lepiej by Tobie, by Tobie siedzieć było w niebie
 	Wsak Twój Tatuś kochany, kochany nie wyganiał Ciebie.
 }
 \stanza{
 	Tam pijałeś coś takie, coś takie słodkie małmazyje,
 	Tu się Twoja gębusia, gębusia łez gorzkich napije.
 }
 \stanza{
 	Tam kukiełki jadałeś, jadałeś z carnuską i miodem,
 	Tu się jeno zasilać, zasilać musis samym głodem.
 }
 \stanza{
 	Tam wciornaska wygoda, wygoda, a tu bieda wsędzie,
 	Ta Ci teraz dokuca, dokuca, ta i potem będzie.
 }
 \stanza{
 	Tam Ty miałeś pościółkę, pościółkę i miętkie piernatki,
 	Tu na to Twej nie stanie, nie stanie ubozuchnej Matki.
 }
 \stanza{
 	Tam Ci zawse słuzyły, słuzyły prześlicne janioły,
 	A tu lezys sam jeden, sam jeden jako palec goły.
 }
 \stanza{
 	Co się więc takiego, takiego Tobie, Panie stało,
 	Zeć się na ten kiepski świat, kiepski świat przychodzić zachciało.
 }
 \stanza{
 	Gdybym ja tam jako Ty, jako Ty tak królował sobie,
 	Nie chciałby ja przenigdy, przenigdy w tym spocywać żłobie.
 }
 \stanza{
 	Albo się więc, mój Panie, mój Panie, wróć do swej krainy,
 	Alboć pozwól się zanieść, się zanieść do nasej chałpiny.
 }
}
\section{W tej kolędzie}{
 \stanza{
 	W tej kolędzie co dziś będzie każdy się ucieszy, 
 	A kto co ma podarować niechaj prędko śpieszy. 
 	} \stanza{
 	Dać dary z tej miary dla Pana małego 
 	By nabył po śmierci zbawienia wiecznego. 
 	} \stanza{
 	Kuba stary przyniósł dary - masła na talerzu, 
 	Sobek parę gołąbeczków takich jeszcze w pierzu. 
 	} \stanza{
 	Wziął Tomek gomółek i jajeczko gęsie, 
 	A Bartek nie miał co dać - dobre chęci niesie.
 	} \stanza{
 	Walek sprawiał tłuste raki nie rychło z wieczora.
 	Nałożywszy dwie kobiele biegł z nimi przez pola.
 	} \stanza{
 	Aż tu strach napotkał Walka nieboraka. 
 	Stanęły dwa wilki niedaleko krzaka. 
 	} \stanza{
 	Gdy obaczył owe gady podskoczył wysoko. 
 	Wielkim strachem przestraszony wybił sobie oko.
 	} \stanza{
 	Uciekał przez krzaki. Podarł se chodaki. 
 	A wilcy mu targali z kobieliny raki. 
 	} \stanza{
 	Szymek wziął kozę na powróz. Prowadzi do Pana. 
 	Śpiewa sobie, wykrzykuje dana moja dana. 
 	} \stanza{
 	Koza się zbrykała, powróz mu urwała. 
 	Skoczywszy jak dzika do lasu bieżała. 
 	} \stanza{
 	On porwawszy się prędziuchno biegł za nią przez krzaki. 
 	Koza skacze jak szalona, spłoszyły ją ptaki. 
 	} \stanza{
 	Uchwycił za ogon trzymając co mocy, 
 	A koza fiknęła, podbiła mu oczy. 
 	} \stanza{
 	Stach kudłaty, chłop bogaty wziął czerwony złoty. 
 	Nie chciał się nikomu kłaniać, biegł prędko do szopy. 
 	} \stanza{
 	Uderzył Jurka w brzuch aż mu kiszki wzruszył, 
 	A Jurko go za łeb, kudły mu ususzył. 
 	} \stanza{
 	Głupi Wojtek nie wziął portek. Mówił: lecej będzie. 
 	Po kolędzie sperki zbierał, gdzie która nabędzie. 
 	 	
 	} \stanza{
 	Biegł Wojtek bez portek po śniegu, po grudzie. 
 	Cieszą się, śmieją się ha ha ha ha ludzie. 
 	} \stanza{
 	Maciek biegł po śliskim lodzie, wybił sobie zęby, 
 	A chciawszy mleko połykać leciało mu z gęby. 
 	} \stanza{
 	Biegł prędko i upadł. Rozbił z mlekiem dzbanek. 
 	Smucił się żałował gdy miał ten trafunek. 
 	} \stanza{
 	Przeto wszyscy oddawajmy temu dary. 
 	Pan to dobry, wszystkim szczodry. Przyjmie nas do chwały. 
 	} \stanza{
 	Niech będzie, niech będzie Jezu pochwalony, 
 	Który jest, który jest w żłobie położony.
 }
}
\section{Wstawszy pasterz bardzo rano}{
 \stanza{
 	Wstawszy pasterz bardzo rano,
 	wyszedł z budy, wlazł na siano,
 	boć go czczyca zejmowała,
 	jaka przedtym nie bywała.
 	} \stanza{
 	Czeka długo, czeka mało,
 	co sie w polu będzie działo:
 	Strach go zewsząd obejmuje,
 	bo śpiewanie z nieba czuje.
 	} \stanza{
 	Porwawszy się poszedł w pole,
 	szukając tam w onym dole,
 	zkąd sie światło podawało,
 	jakie przedtem nie bywało.
 	} \stanza{
 	A tak sobie przechadzając,
 	z podziwieniem rozmyślając,
 	pojrzy wzgórę, aż Anieli
 	pod niebiosy są weseli.
 	} \stanza{
 	Pokój w ziemi ogłaszają,
 	chwalę Bogu powtarzają:
 	że zbawienie oglądało
 	pożądane, wszelkie ciało.
 	} \stanza{
 	Poszedł prędko z tej doliny
 	i od bydła do drużyny,
 	chcąc oznajmić, co sie stało,
 	po północy nim świtało.
 	} \stanza{
 	Lecz ich słyszy z Aniołami
 	krzycząc w polu pod niebami:
 	bieży do nich już weselszy,
 	Azaście tu najmilejsi
 	} \stanza{
 	Braciszkowie moi mili,
 	nie miałem ci takiej chwili,
 	jaka sie tej nocy stała,
 	z czego ziemia radość brała.
 	} \stanza{
 	Panna Syna cudownego
 	porodziła niebieskiego.
 	Róża piękna i lilija
 	Zbawiciela nam powiła.
 	} \stanza{
 	Dziś pasterze wykrzykają,
 	piękne głosy nam wydają,
 	grają w dudki i multanki,
 	drudzy czynią wywijanki.
 	} \stanza{
 	Na kolanach osieł z wołem
 	klęczy przed nim, a my kołem
 	i z muzyką i z pieśniami,
 	z Aniołami, z pastuszkami
 	} \stanza{
 	Uderzamy czołem śmiele,
 	w człowieczem go widząc ciele,
 	żeby przyjął nas do siebie,
 	a po śmierci stawił w niebie.
 }
}
\section{Pasterze mili, coście widzieli}{
 \stanza{
 	Pasterze mili, coście widzieli?
 	Widzieliśmy maleńkiego Jezusa narodzonego,
 	Syna Bożego.
 	} \stanza{
 	Co za pałac miał, gdzie gospodą stał?
 	Szopa bydłu przyzwoita i to jeszcze źle pokryta
 	pałacem była.
 	} \stanza{
 	Jakie łóżeczko, miał Paniąteczko?
 	Marmur twardy, żłób kamienny, na tem depozyt zbawienny
 	spoczywał łożu.
 	} \stanza{
 	Co za obicie miało to Dziecię?
 	Wisząc spod strzech pajęczyna, Boga i Maryi syna
 	obiciem była.} \stanza{
 	W jakiej odzieży Pan nieba leży?
 	Za purpurę, perły drogie ustroiła Go w ubogie
 	pieluszki nędza.} \stanza{
 	Czyli w wygodach, czy spał w swobodach?
 	Na barłogu, ostrem sianie, delikatne spało Panię
 	a nie w łabędziach.} \stanza{
 	Kto asystował? Kto Go pilnował?
 	Wół i osieł przyklękali, parą Go swą zagrzewali
 	dworzanie Jego.} \stanza{
 	Jakie kapele nuciły trele?
 	Aniołowie Mu śpiewali, my na dudkach przygrywali
 	skoczno wesoło.} \stanza{
 	Kto więcej śpieszył, by Dziecię cieszył?
 	Józef stary z Panieneczką za melodyjną piosneczką
 	Dziecię cieszyli.} \stanza{
 	Jakieście dary dali, ofiary?
 	Sercaśmy własne oddali a odchodząc poklękali,
 czołem Mu bili.}
}
\section{Mości gospodarzu}{
 \stanza{
 	Mości gospodarzu, domowy szafarzu,
 	Nie bądź tak ospały, każ nam dać gorzały
 	Dobrej z alembika, i do niej piernika.
 	Hej Kolęda, kolęda.
 }
 \stanza{
 	Chleba pytlowego i masła do niego,
 	Każ stoły nakrywać i talerze zmywać,
 	Każ dać obiad hojny, boś Pan bogobojny.
 	Hej Kolęda, kolęda.
 }
 \stanza{
 	Kaczka do rosołu, sztuka mięsa z wołu,
 	Z gęsi przysmażanie, zjemy to mospanie,
 	I cząber zajęcy i do niego więcej.
 	Hej kolęda, kolęda.
 }
 \stanza{
 	Jędyk do podlewy, Panie miłościwy,
 	I to czarne prosię, pomieści i to się:
 	Każ upiec piecznonki, weźmiem do kieszonki,
 	Hej kolęda, kolęda.
 }
 \stanza{
 	Mości gospodarzu, domowy szafarzu,
 	Każ dać butel wina, bo w brzuchu ruina:
 	Dla większej ofiary, daj dobrej gorzały.
 	Hej kolęda, kolęda.
 }
 \stanza{
 	Piwo będziem pili, będziem się cieszyli,
 	Nie czekaj ruiny, daj połeć słoniny,
 	Dla większej ochoty, daj czerwony złoty.
 	Hej kolęda, kolęda.
 }
 \stanza{
 	Albo talar bity boś Pan wyśmienity,
 	Daj i bóty stare, albo nowych parę,
 	Daj i żupan stary i grosik do fary.
 	Hej kolęda, kolęda.
 }
 \stanza{
 	Mości gospodarzu, domowy szafarzu,
 	Każ spichrze otworzyć i miechy nasporzyć,
 	Zyta ze trzy wory i wołu z obory.
 	Hej kolęda, kolęda.
 }
 \stanza{
 	Na piwo jęczmienia, koni do ciągnienia,
 	Jagły jeśli macie, to nam korzec dacie,
 	Tararki na kaszę, kocham przyjaźń waszę.
 	Hej kolęda, kolęda.
 }
 \stanza{
 	Grochu choć z pół woru, z tutejszego dworu,
 	Na mąkę pszenicy, zjemy społem wszyscy,
 	Owsa ze trzy miary, dla większej ofiary.
 	Hej kolęda, kolęda.
 }
 \stanza{
 	Mościa gospodyni, domowa mistrzyni,
 	Okaż swoją łaskę, każ dać masła faskę,
 	Jeżeliś nie sknęra, daj i kopę sera.
 	Hej kolęda, kolęda.
 }
 \stanza{
 	Mościa gospodyni, domowa mistrzyni,
 	Okaż swoją łaskę, każ upiec kiełbaskę,
 	Którą kiedy zjemy, to podziękujemy.
 	Hej kolęda, kolęda.
 }
}
\section{Hej, w Dzień Narodzenia Syna Jedynego}{
 \stanza{
 	Hej, w Dzień Narodzenia Syna Jedynego
 	Ojca Przedwiecznego, Boga prawdziwego:
 }
 \chorus{
 	Wesoło śpiewajmy,
 	chwałę Bogu dajmy.
 	Hej kolęda! Kolęda!
 }
 \stanza{
 	Panna porodziła niebieskie Dzieciątko,
 	w żłobie położyła małe Pacholątko.
 }
 \chorus{
 	Pasterze śpiewają,
 	na multankach grają.
 	Hej kolęda! Kolęda!
 }
 \stanza{
 	Skoro pastuszkowie o tym usłyszeli,
 	zaraz do Betlejem czem prędzej bieżeli.
 }
 \chorus{
 	Witając Dzieciątko,
 	małe Pacholątko.
 	Hej kolęda! Kolęda!
 }
 \stanza{
 	Kuba nieboraczek nierychło przybieżał,
 	śpieszno bardzo było, wszystkiego odbieżał.
 }
 \chorus{
 	Panu nie miał co dać,
 	kazali mu śpiewać.
 	Hej kolęda! Kolęda!
 }
 \stanza{
 	Dobył tak wdzięcznego głosu baraniego,
 	że się Józef stary przestraszył od niego,
 }
 \chorus{
 	Już uciekać myśli,
 	ale drudzy przyszli.
 	Hej kolęda! Kolęda!
 }
 \stanza{
 	Mówi mu Staruszek: nie śpiewaj tak pięknie,
 	bo się głosu twego Dzieciątko przelęknie,
 }
 \chorus{
 	Lepiej Mu zagrajcie,
 	Panu chwałę dajcie.
 	Hej kolęda! Kolęda!
 }
}
\section{Pójdźmy wszyscy do stajenki}{
 \stanza{
 	Pójdźmy wszyscy do stajenki,
 	Do Jezusa i Panienki,
 	Powitajmy Maleńkiego
 	I Maryję Matkę Jego.
 }
 \stanza{
 	Witaj, Jezu ukochany,
 	Od Patriarchów czekany.
 	Od Proroków ogłoszony,
 	Od narodów upragniony.
 }
 \stanza{
 	Witaj, Dzieciąteczko w żłobie.
 	Wyznajemy Boga w Tobie.
 	Coś się narodził tej nocy.
 	Byś nas wyrwał z czarta mocy.
 }
 \stanza{
 	Witaj Jezu nam zjawiony,
 	witaj dwakroć narodzony,
 	raz z Ojca przed wieków wiekiem
 	a teraz z Matki człowiekiem.
 }
 \stanza{
 	Któż to słyszał takie dziwy?
 	Tyś człowiek i Bóg prawdziwy,
 	Ty łączysz w boskiej osobie
 	dwie natury różne sobie.
 }
 \stanza{
 	O szczęśliwi pastuszkowie,
 	Któż radość Waszą wypowie?
 	Czego ojcowie żądali,
 	Wyście pierwsi oglądali.
 }
 \stanza{
 	Święta Panno, twa przyczyna
 	niech nam wyjedna u Syna,
 	by to jego narodzenie
 	zapewniło nam zbawienie.
 }
}
\section{Triumfy Króla niebieskiego}{
 \stanza{
 	Triumfy Króla niebieskiego
 	zstąpiły z nieba wysokiego.
 	Pobudziły pasterzów,
 	dobytku swego stróżów,
 	śpiewaniem, śpiewaniem, śpiewaniem.
 }
 \stanza{
 	Chwała bądź Bogu w wysokości,
 	a ludziom pokój na niskości.
 	Narodził się Zbawiciel,
 	dusz ludzkich Odkupiciel,
 	na ziemi, na ziemi, na ziemi.
 }
 \stanza{
 	Zrodziła Maryja Dziewica
 	wiecznego Boga bez rodzica.
 	By nas z piekła wybawił,
 	a w niebieskich postawił
 	pałacach, pałacach, pałacach.
 }
 \stanza{
 	Pasterze w podziwieniu stają,
 	triumfu przyczynę badają.
 	Co się nowego dzieje,
 	że tak światłość jaśnieje,
 	nie wiedząc, nie wiedząc, nie wiedząc.
 }
 \stanza{
 	Że to Bóg, gdy się dowiedzieli,
 	swej trzody w polu odbieżeli
 	spiesząc na powitanie
 	do Betlejemskiej stajni
 	Dzieciątka, Dzieciątka, Dzieciątka.
 }
 \stanza{
 	Niebieskim światłem oświeceni
 	pokornie przed nim uniżeni,
 	Bogiem Go być prawdziwym,
 	sercem, afektem żywym
 	wyznają, wyznają, wyznają.
 }
 \stanza{I które mieli z sobą dary
 	Dzieciątku dają na ofiary.
 	Przyjmij o Narodzony,
 	nas i dar przyneisiony
 	z ochotą, z ochotą, z ochotą.
 }
 \stanza{A potem Maryi cześć dają
 	Za Matkę Boską Ją uważają
 	Tak nas uczą przykładem
 	Jak iść mamy ichśladem
 	Statecznie, statecznie, statecznie.
 }
}
\section{Jezus malusieńki}{
 \stanza{
 	Jezus malusieńki, leży wśród stajenki,
 	Płacze z zimna, nie dała Mu Matula sukienki.
 }
 \stanza{
 	Bo uboga była, rąbek z głowy zdjęła,
 	W który Dziecię owinąwszy, siankiem Je okryła.
 }
 \stanza{
 	Nie ma kolebeczki, ani poduszeczki,
 	We żłobie Mu położyła, sianka pod główeczki.
 }
 \stanza{
 	Gdy Dziecina kwili, patrzy każdej chwili,
 	W nóżki zimno, żłóbek twardy, stajenka się chyli.
 }
 \stanza{
 	Matusia truchleje, serdeczne łzy leje:
 	O, mój Synu, wola Twoja, nie moja się dzieje.
 }
 \stanza{
 	Przestań płakać proszę, bo żalu nie zniosę,
 	Dosyć go mam z męki Twojej, którą w sercu noszę.
 }
 \stanza{
 	Pokłon oddawajmy, Bogiem Je wyznajmy,
 	To Dzieciątko ubożuchne ludziom ogłaszajmy.
 }
 \stanza{
 	Niech Je wszyscy znają, serdecznie kochają,
 	Za tak wielkie poniżenie chwałę Mu oddają.
 }
 \stanza{
 	O najwyższy Panie! Waleczny Hetmanie!
 	Zwyciężonyś, mając rączki miłością związane.
 }
 \stanza{
 	Leżysz na tym sianie, Królu nieba, ziemi,
 	Jak baranek na zabicie za moje zbawienie.
 }
 \stanza{
 	Pójdź do serca mego, Tobie otwartego,
 	Przysposób je do mieszkania i wczasu Swojego.
 }
 \stanza{
 	Albo mi daj Swoje, wyrzuciwszy moje,
 	Tak będziesz miał godny pałac na mieszkanie Twoje.
 }
}
\section{Mizerna, cicha}{
 \stanza{
 	Mizerna, cicha, stajenka licha,
 	Pełna niebieskiej chwały.
 	Oto leżący, przed nami śpiący
 	W promieniach Jezus mały.
 }
 \stanza{
 	Przed Nim anieli w locie stanęli
 	I pochyleni klęczą
 	Z włosy złotymi, z skrzydła białymi,
 	Pod malowaną tęczą.
 }
 \stanza{
 	Wielkie zdziwienie: wszelkie stworzenie
 	Cały świat orzeźwiony;
 	Mądrość Mądrości, Światłość Światłości,
 	Bóg - człowiek tu wcielony!
 }
 \stanza{
 	I oto mnodzy, ludzie ubodzy
 	Radzi oglądać Pana,
 	Pełni natchnienia, pełni zbawienia
 	Upadli na kolana.
 }
 \stanza{
 	Oto Maryja, czysta lilija,
 	Przy niej staruszek drżący
 	Stoją przed nami, przed pastuszkami
 	Tacy uśmiechający.
 }
 \stanza{
 	Długo czekali, długo wzdychali,
 	Aż niebo rozgorzało;
 	Piekło zawarte, niebo otwarte
 	Słowo się Ciałem stało
 }
 \stanza{
 	Hej! ludzie prości, Bóg z nami gości,
 	skończony czas niedoli;
 	On daje siebie, chwała na niebie,
 	mir ludziom dobrej woli.
 }
}
\section{A wczora z wieczora}{
 \stanza{
 	A wczora z wieczora,
 	z niebieskiego dwora
 	przyszła nam nowina:
 	Panna rodzi Syna.
 }
 \stanza{
 	Boga prawdziwego,
 	nieogarnionego,
 	za wyrokiem Boskim,
 	w Betlejem żydowskim.
 }
 \stanza{
 	Pastuszkowie mali,
 	w polu w ten czas spali,
 	gdy Anioł w pół nocy
 	światłość z nieba toczy.
 }
 \stanza{
 	Chwałę oznajmując,
 	szopę pokazując,
 	gdzie Panna z Dzieciątkiem,
 	z wołem i oślątkiem.
 }
 \stanza{
 	I Józefem starym,
 	nad Jezusem małym,
 	chwalą Boga swego
 	dziś narodzonego.
 }
 \stanza{
 	Witaj Królu nowy,
 	Synu Dawidowy,
 	Ty nas masz wybawić
 	i w niebie postawić.
 }
}
\section{Wesołą nowinę, bracia, słuchajcie}{
 \stanza{
 	Wesołą nowinę, bracia, słuchajcie.
 	Niebieską Dziecinę ze mną witajcie.
 }
 \chorus{Jak miła ta nowina, mów, gdzie jest ta Dziecina,
 Byśmy tam pobieżeli i ujrzeli.} 
 \stanza{
 	Bogu chwałę wznoszą na wysokości,
 	Pokój ludziom głoszą Duchy światłości.
 }
 \stanza{ 
 	Panna nam powiła Boskie Dzieciątko,
 	Pokłonem uczciła to Niemowlątko.
 }
 \stanza{
 	Którego zrodziła, Bogiem uznała,
 	I Panną jak była, Panną została.
 }
 \stanza{
 	Królowie na wschodzie już to poznali
 	I w judzkim narodzie szukać jechali.
 }
 \stanza{
 	Gwiazda najśliczniejsza ich oświeciła,
 	Do szopy w Betlejem zaprowadziła.
 }
}
\section{Bóg się rodzi}{
\stanza{
    Bóg się rodzi, moc truchleje,
    Pan niebiosów obnażony;
    Ogień krzepnie, blask ciemnieje,
    Ma granice Nieskończony.
}
\chorus{
    Wzgardzony, okryty chwałą;
    Śmiertelny, Król nad wiekami!
	A Słowo Ciałem się stało
	i mieszkało między nami.
}
\stanza{
    Cóż masz niebo nad ziemiany?
    Bóg porzucił szczęście Swoje,
    Wszedł między lud ukochany,
    Dzieląc z nim trudy i znoje.
}
\chorus{
    Niemało cierpiał, niemało,
    Żeśmy byli winni sami!
	A Słowo...
}
\stanza{
    W nędznej szopie urodzony,
    Żłób Mu za kolebkę dano!
    Cóż jest? Czym był otoczony?
    Bydło, pasterze i siano!
}
\chorus{
    Ubodzy, was to spotkało,
    Witać Go przed bogaczami!
	A Słowo...
}
\stanza{
    Podnieś rękę, Boże Dziecię,
    Błogosław Ojczyznę miłą,
    W dobrych radach, w dobrym bycie
    Wspieraj jej siłę swą siłą.
}
\chorus{
    Dom nasz i majętność całą,
    I wszystkie wioski z miastami!
	A Słowo... 
}
\stanza{
    Potem i króle widziani
    Cisną się między prostotą,
    Niosąc dary Panu w dani:
    Mirrę, kadzidło i złoto.
}
\chorus{
	Bóstwo to razem zmieszało
    Z wieśniaczymi ofiarami!
	A Słowo...
}
}
\section{Dzisiaj w Betlejem}{
\stanza{
    Dzisiaj w Betlejem, dzisiaj w Betlejem wesoła nowina,
    Że Panna czysta, że Panna czysta porodziła Syna.
}
\chorus{
    Chrystus się rodzi, nas oswobodzi,
    Anieli grają, króle witają,
    Pasterze śpiewają, bydlęta klękają,
    Cuda, cuda ogłaszają.
}
\stanza{
    Maryja Panna, Maryja Panna Dzieciątko piastuje,
    I Józef święty, i Józef święty Ono pielęgnuje.
}
\stanza{
    Choć w stajeneczce, choć w stajeneczce Panna syna rodzi,
    Przecież On wkrótce, przecież On wkrótce ludzi oswobodzi.
}
\stanza{
    I Trzej Królowie, i Trzej Królowie od wschodu przybyli,
    I dary Panu, i dary Panu kosztowne złożyli.
}
\stanza{
    Pójdźmy też i my, pójdźmy też i my przywitać Jezusa,
    Króla na króle, Króla nad króle uwielbić Chrystusa.
}
\stanza{
    Bądźże pochwalon, bądźże pochwalon dziś, nasz wieczny Panie,
    Któryś złożony, któryś złożony na zielonym sianie.
}
\stanza{
    Bądź pozdrowiony, bądź pozdrowiony Boże nieskończony,
    Sławimy Ciebie, sławimy Ciebie, Jezu niezmierzony.
}
}
\section{Przybieżeli do Betlejem pasterze}{
 \stanza{
 	Przybieżeli do Betlejem pasterze,
 	Grając skocznie Dzieciąteczku na lirze.
 }
 \chorus{
 	Chwała na wysokości,
 	Chwała na wysokości,
 	A pokój na ziemi.
 }
 \stanza{
 	Oddawali swe ukłony w pokorze
 	Tobie z serca ochotnego, o Boże!
 }
 \stanza{
 	Anioł Pański sam ogłosił te dziwy,
 	Których oni nie słyszeli, jak żywi.
 }
 \stanza{
 	Dziwili się napowietrznej muzyce
 	I myśleli, co to będzie za Dziecię?
 }
 \stanza{
 	Oto mu się wół i osioł kłaniają,
 	Trzej królowie podarunki oddają.
 }
 \stanza{
 	I anieli gromadami pilnują
 	Panna czysta wraz z Józefem piastują.
 } 
 \stanza{
 	Poznali Go Mesyjaszem być prawym
 	Narodzonym dzisiaj Panem łaskawym
 }
 \stanza{
 	My go także Bogiem, Zbawcą już znamy
 	I z całego serca wszyscy kochamy.
 }
}
\section{Wśród nocnej ciszy}{
\stanza{
    Wśród nocnej ciszy głos się rozchodzi:
    Wstańcie, pasterze - Bóg się wam rodzi!
}
\chorus{
    Czem prędzej się wybierajcie,
    Do Betlejem pospieszajcie
    Przywitać Pana.
    Przywitać Pana.
}
\stanza{
    Poszli, znaleźli Dzieciątko w żłobie,
    Z wszystkimi znaki danymi sobie.
}
\chorus{
    Jako Bogu cześć Mu dali,
    A witając zawołali,
    Z wielkiej radości.
    Z wielkiej radości.
}
\stanza{
    Ach, witaj Zbawco, z dawna żądany!
    Tyle tysięcy lat wyglądany;
}
\chorus{
    Na Ciebie króle, prorocy
    Czekali, a Tyś tej nocy
    Nam się objawił.
    Nam się objawił.
}
\stanza{
    I my czekamy na Ciebie, Pana,
    A skoro przyjdziesz na głos kapłana,
}
\chorus{
    Padniemy na twarz przed Tobą,
    Wierząc, żeś jest pod osobą
    Chleba i wina.
    Chleba i wina.
}
}